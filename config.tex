% ----------------------------------------------- github
% PACKAGES

\usepackage{arxiv}
\usepackage{charter}
\usepackage{fullpage}

% TODOS
\usepackage{todonotes}

% DEFINITIONS
\usepackage{xparse}
\usepackage{xargs} % define commands with multiple arguments with default a value http://www.texfaq.org/FAQ-twooptarg

% ALGORITHM PSEUDOCODE
% Wrapper for pseudocode
\usepackage{algorithm}
%\usepackage{algpseudocode}
\usepackage{algcompatible}

% to insert code and define style of your snippets
\usepackage{listings}
% colors for listings
\definecolor{codebg}{rgb}{0.95,0.95,0.92}
\definecolor{keyword}{rgb}{0.80,0.00,0.50}   % magenta-ish
\definecolor{string}{rgb}{0.58,0.00,0.82}    % purple
\definecolor{comment}{rgb}{0.00,0.45,0.00}   % dark green
\definecolor{linenums}{rgb}{0.50,0.50,0.50}  % grey
% custom style for listings
\lstdefinestyle{mystyle}{
  language       = Python,
  basicstyle     = \ttfamily\small,
  backgroundcolor= \color{codebg},
  keywordstyle   = \color{keyword}\bfseries,
  stringstyle    = \color{string},
  commentstyle   = \color{comment}\itshape,
  emph={len,math,ceil,merge},      % <-- new
  emphstyle=\color{blue},          % <-- new
  numberstyle    = \tiny\color{linenums},
  numbers        = left,
  showstringspaces = false,
  breaklines       = true,
  tabsize          = 4
}
% set `mystyle` as default style for all lstlisting
\lstset{style=mystyle}




% MISC
\usepackage[utf8]{inputenc}
\usepackage[T1]{fontenc}
\usepackage{textcomp}
\usepackage{lmodern}
\usepackage{babel}
\usepackage{lipsum}
\usepackage{graphicx}
\usepackage{scalerel,stackengine}
\usepackage{subcaption}
\usepackage{tikz}
\usetikzlibrary{positioning, calc}
\usepackage{datetime}

\usepackage{accents}
\usepackage{xcolor,color}
\usepackage{parskip}
\usepackage{amssymb,amsmath,amsthm,amsfonts,mathtools,bm,bbm, centernot} % negate any symbol, e.g., \centernot\implies
\usepackage[colorlinks=true, urlcolor=blue, linktoc = all]{hyperref}
\usepackage{thmtools}
\usepackage{stmaryrd}
\usepackage{multirow}
\usepackage{booktabs}
\usepackage{alphabeta} % Allows \alpha, etc. in text mode.
\usepackage{thm-restate}
\usepackage{nicefrac}
\usepackage[square,authoryear]{natbib}
\let\cite\citep   % make \cite --> [Author, year]


\usepackage[capitalise,nameinlink,noabbrev]{cleveref}

% Font for math and code
% \usepackage[scaled=0.95]{FiraMono}
\usepackage{libertinus}
\usepackage{libertinust1math}










% -----------------------------------------------
% TODOS
\newif\ifTODO 

% \TODOtrue
%\ifanonsubmission \TODOfalse \fi
\TODOfalse

\definecolor{aquamarine}{rgb}{0.5, 1.0, 0.83}

\newcommand{\Gab}[1]{\ifTODO \todo[inline,author=GI, caption={}, color=aquamarine]{#1} \fi}

\newcommand{\gab}[1]{\ifTODO \todo[size=\scriptsize, color=aquamarine]{GI: #1}{} \fi}

\newcommand{\gabrev}[1]{\textcolor{black}{#1}}
\newcommand{\warn}[1]{\textcolor{red}{\textbf{#1}}}
\newcommand{\cmd}[1]{\textcolor{teal}{\textbf{\texttt{#1}}}}
\newcommand{\bfu}[1]{\textbf{\underline{#1}}}


\ifTODO
% Make the paper wider if ifTODO is set so that the notes fit
\paperwidth=\dimexpr \paperwidth + 6cm\relax
\oddsidemargin=\dimexpr\oddsidemargin + 3cm\relax
\evensidemargin=\dimexpr\evensidemargin + 3cm\relax
\marginparwidth=\dimexpr \marginparwidth + 3cm\relax
\fi










% -----------------------------------------------
% DEFINITIONS
\newcommand{\PL}{\newlink{def:acronym_polyak_lojasiewicz}{PL}}
\newcommand{\QG}{\newlink{def:acronym_quadratic_growth}{QG}}
\newcommand{\FW}{\newlink{def:acronym_frank_wolfe}{FW}}
\newcommand{\AFW}{\newlink{def:acronym_away_frank_wolfe}{AFW}}
\newcommand{\LMO}{\newlink{def:acronym_linear_minimization_oracle}{LMO}}

\newcommand{\w}[1]{\operatorname{\delta}_{#1}}
\newcommand{\what}{\operatorname{\bar{\delta}}}
\newcommand{\X}{\mathcal{X}}
\newcommand{\wface}[2]{\operatorname{\Delta}_{#1}(#2)}
\newcommand{\whatface}[2]{\operatorname{\bar{\Delta}}_{#1}(#2)}
\newcommand{\bigomega}[1]{\Omega( #1 )}
\newcommand{\bigomegal}[1]{\Omega\left( #1 \right)}
\newcommand{\bigo}[1]{O( #1 )}
\newcommand{\bigol}[1]{O\left( #1 \right)}

% these two lines make sure that \cref{eq:eq_14} gives "Eq.~(14)"
%\crefname{equation}{}{} % you can \cref an equation and you only get (X)
\crefname{equation}{Eq.}{Eqs.}
% sets
\newcommand{\mP}{\mathcal{P}}
\newcommand{\mQ}{\mathcal{Q}}
\newcommand{\mA}{\mathcal{A}}
\newcommand{\mF}{\mathcal{F}}
\newcommand{\mX}{\mathcal{X}}
\newcommand{\verts}[1]{\mathcal{V}_{#1}} % vertices of a polyhedron
\newcommand{\conv}[1]{\mathsf{conv}\left(#1\right)} % convex hull
\newcommand{\aff}[1]{\mathsf{aff}\left(#1\right)} % affine hull
\newcommand{\faces}[1]{{\mathsf{faces}^\ast}\left(#1\right)} % faces
\newcommand{\facets}[1]{\mathsf{facets}\left(#1\right)} % facets
\newcommand{\dist}[1]{\mathsf{dist}\left(#1\right)} % distance function
\newcommand{\proj}{\mathsf{Proj}} % projection operator
\newcommand{\transpose}[1]{#1^\top} % transpose matrix
\newcommand{\id}{\mathbf{I}} % identity matrix
% option available to write the following constants with an exponent. Use: pw[2]{X}
\NewDocumentCommand{\pw}{o m}{%
  \mathop{%
    \delta%
    \IfValueT{#1}{^{#1}}%
    _{#2}%
  }%
}
\NewDocumentCommand{\vf}{o m}{%
  \mathsf{vf}%
  \IfValueT{#1}{^{#1}}%
  _{#2}%
}
% diameter
\NewDocumentCommand{\diam}{o m}{
  D
  \IfValueT{#1}{^{#1}}_#2
}
% expected value operator with optional superscript and subscript
% plain \ev{X} expected value of X
% superscript only
% \ev[2]{X^2} (expected value squared)
% subscript only
% \ev[][Y]{\proba{X | Y}} (expected value wrt Y)
% with both
% \ev[2][Y]{g(X,Y)}
\NewDocumentCommand{\ev}{o o m}{
	\mathop{}\!\mathbb{E}
	\IfValueT{#1}{^{#1}}
	\IfValueT{#2}{_{#2}}
	\left[ #3 \right]
	
	
}% variance operator
\NewDocumentCommand{\var}{o m}{%
  \mathop{}\!\operatorname{Var}%
  \IfValueT{#1}{^{#1}}%
  \left[ #2 \right]%
}
% covariance operator (comma-style argument)
\NewDocumentCommand{\cov}{o m}{%
  \mathop{}\!\operatorname{Cov}%
  \IfValueT{#1}{^{#1}}%
  \left[ #2 \right]%
}

% mutual information operator
\NewDocumentCommand{\mi}{o m}{%
  \mathop{}\!\operatorname{MI}%
  \IfValueT{#1}{^{#1}}%
  \left( #2 \right)%
  
% probability operator
\NewDocumentCommand{\proba}{o m}{%
	\mathop{}\!\mathbb{P}%
	\IfValueT{#1}{^{#1}}%
	\left( #2 \right)%
}
% probability distribution function
% Usage \PDF[-1]{X}[x], arguments in square brackets are optionale
\NewDocumentCommand{\PDF}{O{} m o}{%
  f_{#2}%
  \IfValueT{#1}{^{#1}}%
  \IfNoValueTF{#3}{}{%
    \!\left(#3\right)%
  }%
}
% cumulative distribution function
\NewDocumentCommand{\CDF}{O{} m o}{%
  \Phi_{#2}%
  \IfValueT{#1}{^{#1}}%
  \IfNoValueTF{#3}{}{%
    \!\left(#3\right)%
  }%
}
% probability density function
\NewDocumentCommand{\PMF}{O{} m o}{%
  p_{#2}%
  \IfValueT{#1}{^{#1}}%
  \IfNoValueTF{#3}{}{%
    \!\left(#3\right)%
  }%
}





% inner product
\newcommand{\innp}[1]{\left\langle #1 \right\rangle}
% rank operator
\newcommand{\rank}[1]{\operatorname{rank}\left(#1\right)}

\DeclareMathOperator*{\argmin}{\mathrm{arg\,min}}
\DeclareMathOperator*{\argmax}{\mathrm{arg\,max}}
\DeclarePairedDelimiter\ceil{\lceil}{\rceil}
\DeclarePairedDelimiter\floor{\lfloor}{\rfloor}
\newcommand{\norm}[1]{\| #1 \|} 
\newcommand{\defi}{\stackrel{\mathrm{\scriptscriptstyle def}}{=}}
\newcommand{\defiin}{\stackrel{\mathrm{\scriptscriptstyle def}}{\in}}

\newtheorem{theorem}{Theorem}[section]
\newtheorem{lemma}[theorem]{Lemma}
\newtheorem{remark}[theorem]{Remark}
\newtheorem{proposition}[theorem]{Proposition}
\newtheorem{definition}[theorem]{Definition}
\newtheorem{corollary}[theorem]{Corollary}
\newtheorem{fact}[theorem]{Fact}
\newtheorem{example}[theorem]{Example}
\newtheorem{assumption}[theorem]{Assumption}
\newtheorem{exercise}[theorem]{Exercise}
\newenvironment{solution}%
{%
  \par\noindent{\bfseries\upshape Solution\ }%
}%
{%
  \jmlrQED
}



\newcommand\equalhat{\mathrel{\stackon[1.5pt]{=}{\stretchto{%
    \scalerel*[\widthof{=}]{\wedge}{\rule{1ex}{3ex}}}{0.5ex}}}}

% Usage: \circled{1}[\leq]
\newcommand*\circledaux[1]{\tikz[baseline=(char.base)]{
    \node[shape=circle,draw,inner sep=0.8pt] (char) {#1};}}

\NewDocumentCommand{\circled}{ m o }{%
    \IfNoValueTF{#2}{ \circledaux{#1} }{ \stackrel{\circledaux{#1}}{#2} }%
}







% -----------------------------------------------
% HYPERLINK SETUP
\hypersetup{
    colorlinks=true,
    linkcolor=blue,
    filecolor=magenta,
    urlcolor=blue,
    citecolor=blue
}










% -----------------------------------------------
% PROOFS
\makeatletter

\def\Hy@raisedlink@left#1{%
    \ifvmode
        #1%
    \else
        \Hy@SaveSpaceFactor
        \llap{\smash{% \llap{ added
        \begingroup
            \let\HyperRaiseLinkLength\@tempdima
            \setlength\HyperRaiseLinkLength\HyperRaiseLinkDefault
            \HyperRaiseLinkHook
        \expandafter\endgroup
        \expandafter\raise\the\HyperRaiseLinkLength\hbox{%
            \Hy@RestoreSpaceFactor
            #1%
            \Hy@SaveSpaceFactor
        }%
        }}% } to close \llap
        \Hy@RestoreSpaceFactor
        \penalty\@M\hskip\z@ 
    \fi
}

									 
													 
\newcommand\newlink[2]{\hyperlink{#1}{\normalcolor #2}}
\makeatletter
\newcommand\newtarget[2]{\Hy@raisedlink{\hypertarget{#1}{}}#2}
\makeatother

% -----------------------------------
% Links between proposition or theorem, and its proof
\newcommand\linktoproof[1]{{\normalfont[{\hyperlink{proof:#1}{$\downarrow$}}]}}
\newcommand\linkofproof[1]{\textbf{of \cref{#1}. }\newtarget{proof:#1}}

% -----------------------------------
% Links between exercise and its solutions
\newcommand\linktosol[1]{%
  \label{#1}%
  {\normalfont[{\hyperlink{sol:#1}{$\downarrow$}}]}%
}
\newcommand\linkofsol[1]{\textbf{of \cref{#1}. }\newtarget{sol:#1}}

% -----------------------------------
% new QED square
\newcommand{\jmlrBlackBox}{\rule{1.5ex}{1.5ex}}
\providecommand{\BlackBox}{\jmlrBlackBox}
\newcommand{\jmlrQED}{\hfill\jmlrBlackBox\par\bigskip}
\providecommand{\proofname}{Proof}
\renewenvironment{proof}%
{%
 \par\noindent{\bfseries\upshape \proofname\ }%
}%
{\jmlrQED}










% -----------------------------------------------
% ALGORITHM PSEUDOCODE
% Config caption names:
%\renewcommand{\lstlistingname}{Algorithm}
\algnewcommand{\lst}{\texttt{lst}}
\algnewcommand{\slst}{\texttt{slst}}
\algnewcommand{\SEND}{\textbf{send}}

\newsavebox{\algleft}
\newsavebox{\algright}

\renewcommand{\algorithmiccomment}[1]{\hfill  $\diamond$ #1}

\renewcommand{\algorithmicrequire}{\textbf{Input:}}
\renewcommand{\algorithmicensure}{\textbf{Output:}}

\makeatletter
\newcounter{algorithmicH}% New algorithmic-like hyperref counter
\let\oldalgorithmic\algorithmic
\renewcommand{\algorithmic}{%
  \stepcounter{algorithmicH}% Step counter
  \oldalgorithmic}% Do what was always done with algorithmic environment
% \renewcommand{\theHALG@line}{ALG@line.\thealgorithmicH.\arabic{ALG@line}}
\makeatother


% ========================================================
\makeatletter
\newenvironment{breakablealgorithm}
  {% \begin{breakablealgorithm}
   \begin{center}
     \refstepcounter{algorithm}% New algorithm
     \hrule height.8pt depth0pt \kern2pt% \@fs@pre for \@fs@ruled
     \renewcommand{\caption}[2][\relax]{% Make a new \caption
       {\raggedright\textbf{\ALG@name~\thealgorithm} ##2\par}%
       \ifx\relax##1\relax % #1 is \relax
         \addcontentsline{loa}{algorithm}{\protect\numberline{\thealgorithm}##2}%
       \else % #1 is not \relax
         \addcontentsline{loa}{algorithm}{\protect\numberline{\thealgorithm}##1}%
       \fi
       \kern2pt\hrule\kern2pt
     }
  }{% \end{breakablealgorithm}
     \kern2pt\hrule\relax% \@fs@post for \@fs@ruled
   \end{center}
  }

%\newcommand{\theHalgorithm@line}{\thealgorithm.\arabic{algorithm@line}}

\makeatother
